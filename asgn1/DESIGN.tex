\documentclass[11pt]{article}
\usepackage{graphicx}
\usepackage{fullpage}
\usepackage{fourier}
\usepackage{xspace}
\usepackage{booktabs}
\usepackage{wrapfig}

\title{Assignment 1 DESIGN.pdf}
\author{Victor Nguyen}

\begin{document}\maketitle

\section{Description of Program:}
This bash script uses the collatz.c file (provided by the professor) to produce 3 distinct graphs
about the collatz sequence.


\section{Files to be included in directory "asgn1":}

\begin{enumerate}
    \item
    plot.sh
    \begin{itemize}
        \item
        Script to produce the 3 collatz graphs
    \end{itemize}

    \item
    collatz.c
    \begin{itemize}
        \item
        The c code that generates any given collatz sequence if specified.
    \end{itemize}

    \item
    Makefile
    \begin{itemize}
        \item
        A file that can execute shell commands (makes it easier to clean up and build other files)
    \end{itemize}

    \item
    README.md
    \begin{itemize}
        \item
        Text file in markdown format that describes how to build and run the script. 
    \end{itemize}

    \item
    DESIGN.tex
    \begin{itemize}
        \item
        .tex file that contains the source code of this .pdf file.
    \end{itemize}

    \item
    WRITEUP.tex
    \begin{itemize}
        \item
        .tex file that contains the source code for the WRITEUP.pdf file.
    \end{itemize}
\end{enumerate}


\section{Pseudocode / Structure:}

To begin generating the collatz sequence lengths, we need a for loop that starts at the number 
2 and iterates to the value 10000. Inside the for loop, I need to run the collatz c program 
and generate the current collatz sequence in the iteration. With that I want to count the lines 
generated from this specific collatz sequence and store this value inside a temporary .dat file 
(along side with the iteration number). This .dat file should contain the iteration number on 
the x axis and the length of each number on the y axis.

\noindent For the second graph, where we need to find the maximum collatz sequence value, my idea
is to create another for loop that starts at 2 and iterates to 10000. Inside the for loop, I would 
need to run the collatz program and sort the numbers numerically. When it's sorted I want to grab 
the highest value, contained in that file and concatenate that number with the iteration number 
into another file that will be used as data for the plot.

\noindent The third graph, which wants a histogram of the collatz sequence length reoccurrences, 
will require another for loop that starts at 2 and iterates to 10000. Inside the for loop, I need 
to run collatz program to generate the sequences of each number, sort the values, and grab the 
highest value in each iteration and store it into a file. Afterwards, I need to sort this new file 
and count the occurrences of the same number appearing each time, and store it into a .dat file that would be used as data points for the third graph.

\section{Error Handling:}

I used many of the example files that were provided by the class through the resources repo. 
Learning syntax was the main culprit when I ran into errors. I debugged and read through manuals 
to fix most of those problems.

\section{Credit:}

The asgn1.pdf that was provided by the professor had some example codes that really helped me. 
Particularly the sort, awk, and uniq commands helped during the process.

\noindent I attended Eugene's section where he helped answer a few questions of mine regarding 
syntax. He also helped me understand Makefiles, the collatz.c program, and scripting. 

\end{document}
