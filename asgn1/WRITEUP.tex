\documentclass[11pt]{article}
\usepackage{graphicx}
\usepackage{fullpage}
\usepackage{fourier}
\usepackage{xspace}
\usepackage{booktabs}
\usepackage{wrapfig}

\title{Assignment 1 - WRITEUP.pdf}
\author{Victor Nguyen}

\begin{document}\maketitle

\section{Plots and Analysis}
In this assignment, I was tasked to learn UNIX commands on the provided collatz.c file. The 
collatz.c file is the source code that generates a random collatz sequence (or a specific sequence 
if you know the proper syntax for it). Anyways, here are some of the graphs I produced with the 
provided source code.

\begin{center}
\includegraphics{collatz.pdf}
\end{center}

\noindent This first graph depicts the relationship between a given collatz sequence n, and it's 
length. Commands I've used to produce this plot include for loops, executing collatz, echo, wc, and 
cat. Here are some reasons why:

\begin{itemize}
    \item
    For loop - Needed to generate all collatz sequences from 2 to 10000.
    \item
    ./collatz - Needed to use the source code to generate the collatz sequences.
    \item
    echo - Needed to know what sequence number I was on and to put that on the graph.
    \item
    wc - Used to count the number of lines of a given sequence number.
    \item
    cat - Concatenate the length of the sequence number alongside the sequence number.
\end{itemize}

\noindent
I've also used the gnuplot utility to take in a .dat file that I generated through this script, 
and use those data points to graph through gnuplot. Most of the gnuplot specifications were 
provided by the professor.

\noindent
Analyzing the graph, we can see that the growth of the sequence lengths gets smaller when we 
iterate to higher collatz sequences. It's growth is similar to the square root of x (if graphed). 


\begin{center}
\includegraphics{collatz_max.pdf}
\end{center}

\noindent
This next graph depicts the relationship between a given collatz sequence n, and the highest value 
that the given sequence could rise to. Commands I've used to produce this graph include for loops, 
collatz, sort, head, echo, and cat. Some of these commands have the same usage as mentioned before, 
so to reduce redundancy I will only mentioned the new commands and their usage below.

\begin{itemize}
    \item
    sort - Needed to sort the collatz sequence numerically so I can find the highest value easily.
    \item
    head - Used to grab the highest value in the given collatz sequence.
\end{itemize}

\noindent
For the gnuplot, the only thing I changed was fixing the y axis to only look at numbers 0 to 100000.

\noindent
Analyzing the graph, for the most part, the growth of some numbers are linear. There are a few 
trends where some collatz sequences ending up with the same maximum value, producing a visible 
horizontal lines on the graph.


\begin{center}
\includegraphics{collatz_hist.pdf}
\end{center}

\noindent
For the last graph, it depicts the relationship between the lengths of the collatz sequences (from 
2 to 10000) and how many of those sequences match the same length of other sequences. Commands I've 
used to produce this graph include for loops, collatz, wc, sort, uniq, and awk. 

\begin{itemize}
    \item
    uniq - Used to count the occurrences of the same number.
    \item
    awk - Used to modify the .dat file and flip the x and y values.
\end{itemize}

\noindent
For the gnuplot, I had to specify the x axis and also change the x axis tics to get the desired 
graph.

\noindent
Analyzing the graph, I noticed a high density of sequences with the lengths around 50. There also 
seems to be another peak in the 125 to 150 range. The frequency ranges tells me that most collatz 
sequences end within 200 "steps", and rarely last above that. This gives us slight evidence of 
collatz sequences always following that same 4 2 1 sequence.

\section{Conclusion}

From the data that we collected on the collatz sequences (ranging 2 to 10000).
We can conclude that:

\begin{enumerate}
    \item
    The higher the collatz sequence gets, the proportion between the length of the sequence "l" 
and that the value "v" increases, i.e. l:v over time becomes l:v + time.
    \item
    The collatz sequence values grow in a somewhat, linear fashion. 
    \item
    Collatz sequences tend to only last around 200 steps before falling into the 4 2 1 sequence.
\end{enumerate}
\end{document}
